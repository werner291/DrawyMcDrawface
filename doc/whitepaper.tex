\documentclass[11pt]{article}

\usepackage{fullpage}

%Gummi|065|=)
\title{Drawing a scene described through natural language}
\author{Werner Kroneman}
\date{}
\begin{document}

\maketitle

\section{Introduction}

\section{A use cases}

The algorithm will allow for iterative changes tot he scene being described.

A common scenario will look somewhat like the this:

\begin{enumerate}

\item The user starts the program

\item The user enters "Create 15 pineapples"

\item The program places 15 pineapples on the screen

\item The user enters "The pineapples are pink"

\item The program makes the pineapples pink instead of yellow

\item The user enters "Make 5 of the pineapples a bit larger than the others"

\item The program makes 5 of the pineapples a bit larger

\end{enumerate}

From this, we will make some observations:

\begin{itemize}

\item The program needs to know how to draw a pineapple

\item While pineapples are usually yellow, the program should be able to make them pink.

\item Previous commands are important. For example, suppose there are many pineapples in the scene. If the user refers to "the pineapples", we need a way to know which ones they refer to.

\end{itemize}

\section{The algorithm}

The algorithm will work in an iterative manner, roughly according to the following scheme:

\begin{enumerate}

\item Interpret the commands by the user

\item Translate these commands into an abstract, high-level description of the scene.

\item Translate the high-level description into something that can be interpreted in a somewhat more straightforward manner

\item Use that result to draw a scene

\end{enumerate}

\subsection{Interpreting the commands}

Commands are simple, correct English sentences such as "Add 5 lions.". Correcting grammar and spelling is beyond the scope of this project.

Next, these commands are sent through Google's ParseyMacParseface, the result of which is a tree that describes the sentence as a tree that provides sentence analyis.

Then, the parsed commands are classified into one of the following categories:

\begin{itemize}
	\item Creation: this command involves addition of some item into the scene.
	\item Change: this command applies a change to some existing item in the scene.
	\item Deletion: this command involves removing some item from the scene.
\end{itemize}

\end{document}
